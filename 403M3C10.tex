\documentclass{beamer}

\usepackage{amsmath,amsthm,amsfonts}
\usepackage{mathrsfs,mathtools,mathdots}
\usepackage{caption,subcaption}
\usepackage{pgf,tikz}
\usepackage{pgfplots}
\usepackage{tikz-3dplot}
\usetikzlibrary{shapes.geometric,arrows,arrows.meta,decorations.pathreplacing,bending}

\title{Differential Geometry}
\author{Module III}
\institute{Chapter 10 : The Curvature of Plane Curves}

\begin{document}

\begin{frame}
\maketitle
\end{frame}

\begin{frame}{Curvature}
	In $\mathbb{R}^2$,  
	$$ \text{ Weingarten map }L_p : C_p \to C_p \text{ is }L_p(\boldsymbol{v}) = \kappa(p) \boldsymbol{v}$$
	\vfill
	Curvature of plane curve $C$ in $\mathbb{R}^2$ is
	$$ \kappa : \mathbb{R}^2 \to \mathbb{R},\ \kappa(p) = L_p(\boldsymbol{v}) \cdot \boldsymbol{v}/\| \boldsymbol{v}\|^2 $$
	\vfill
	It turns out that, the curvature is a measure of the normal component of accelation,
	$$ \kappa(\alpha(t)) = \frac{\ddot{\boldsymbol{\alpha}}(t) \cdot \boldsymbol{N}(\alpha(t))}{\| \dot{\boldsymbol{\alpha}}(t)\|^2} $$
\end{frame}

\begin{frame}
	\vspace{0.6in}
	\hspace{3cm} {\color{blue}\Huge{Thank You}}
\end{frame}
\end{document}
