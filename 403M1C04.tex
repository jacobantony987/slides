\documentclass{beamer}

\usepackage{pgf,tikz}
\usepackage{pgfplots}
\usepackage{tikz-3dplot}
\usetikzlibrary{shapes.geometric,arrows.meta,decorations.pathreplacing}

\title{Differential Geometry}
\author{Module I}
\institute{Chapter 4 : Surfaces}

\begin{document}

\begin{frame}
\maketitle
\end{frame}

\begin{frame}{$n$-Surface}
\begin{definition}[$n$-Surface]
	A non-empty subset $S$ of $\mathbb{R}^{n+1}$ is an $n$-surface if it is of the form $S = f^{-1}(c)$ where $f : U \to \mathbb{R}$, $U$ open subset of $\mathbb{R}^{n+1}$ is a smooth function with the property that $\nabla f(p) \ne 0$ for every $p \in S$.
\end{definition}
\begin{description}
	\item[$n=1$] plane curve
	\item[$n=2$] surface
	\item[$n>2$] hypersurface
\end{description}
The $n$-surfaces are subspaces of dimension $n$.
\end{frame}

\begin{frame}{Summary : Surface}
\begin{definition}[Surface]
\begin{itemize}
	\item The subset $S$ of the level set of a smooth function $f$
	\item The points of the surface $S$ are regular points of $f$
	\item Depends on $S$, but independent of the function $f$
\end{itemize}
\end{definition}
\begin{block}{Why independent of $f$ ?}
\begin{itemize}
	\item Level sets are independent of the function
	\item Different level sets can have same subset
		%Level set of different functions could be the same. Again, the surface could be subset of different level sets.
\end{itemize}
\end{block}
\end{frame}

\begin{frame}{Example : $n$-sphere}
\begin{definition}[$n$-sphere]
	The unit $n$-sphere $x_1^2 + x_2^2 + \cdots + x_{n+1}^2 = 1$ is the level set $f^{-1}(1)$ of the function $f:\mathbb{R}^{n+1} \to \mathbb{R}$ defined by
\begin{equation}
	f(x_1,x_2,\cdots,x_{n+1}) = x_1^2+x_2^2+\cdots+x_{n+1}^2
\end{equation}
\end{definition}
\end{frame}

\begin{frame}{$n$-Sphere as level set}
	$$f(x_1,x_2,\cdots,x_{n+1}) = x_1^2+x_2^2+\cdots+x_{n+1}^2$$
\begin{align*}
	f^{-1}(1) & = \{(x_1,x_2,\cdots,x_{n+1}) : f(x_1,x_2,\cdots,x_{n+1}) = 1\} \\
	& = \{ (x_1,x_2,\cdots,x_{n+1}) : x_1^2 + x_2^2 + \cdots x_{n+1}^2 = 1\}
\end{align*}
	This level set is the set of all points in $\mathbb{R}^{n+1}$ satsifying 
\begin{equation}
	x_1^2 + x_2^2 + \cdots x_{n+1}^2 = 1
\end{equation}
\end{frame}

\begin{frame}{$n$-Spheres}
	The $n$-Sphere is an $n$-dimensional surface in $\mathbb{R}^{n+1}$.
\begin{description}
	\item[$n=1$] unit circle
		$$x_1^2 + x_2^2 = 1$$
	\item[$n=2$] unit sphere
		$$x_1^2 + x_2^2 + x_3^2 = 1$$
	\item[$n>2$] hypersphere
		$$x_1^2 + x_2^2 + \cdots + x_{n+1}^2 = 1$$
\end{description}
\end{frame}

\begin{frame}{Example : $n$-plane}
\begin{definition}[$n$-plane]
	An $n$-plane $a_1x_1 + a_2x_2 + \cdots + a_{n+1}x_{n+1} = b$ is the level set $f^{-1}(b)$ of the function $f: \mathbb{R}^{n+1}\to \mathbb{R}$ defined by
\begin{equation}
	f(x_1,x_2,\cdots,x_{n+1}) = a_1x_1 + a_2x_2 + \cdots + a_{n+1}x_{n+1}
\end{equation}
\end{definition}
\end{frame}

\begin{frame}{$n$-Plane as level set}
	$$f(x_1,x_2,\cdots,x_{n+1}) = a_1x_1 + a_2x_2 + \cdots + a_{n+1}x_{n+1}$$
\begin{align*}
	f^{-1}(b) & = \{(x_1,x_2,\cdots,x_{n+1}) : f(x_1,x_2,\cdots,x_{n+1}) = b\} \\
	& = \{ (x_1,x_2,\cdots,x_{n+1}) : a_1x_1 + a_2x_2 + \cdots a_{n+1}x_{n+1} = b\}
\end{align*}
	This level set is the set of all points in $\mathbb{R}^{n+1}$ satsifying 
\begin{equation}
	a_1x_1 + a_2x_2 + \cdots a_{n+1}x_{n+1} = b
\end{equation}
\end{frame}

\begin{frame}{$n$-Planes}
	The $n$-Plane is an $n$-dimensional surface in $\mathbb{R}^{n+1}$.
\begin{description}
	\item[$n=1$] line in $\mathbb{R}^2$
		$$a_1x_1 + a_2x_2 = b$$
	\item[$n=2$] plane in $\mathbb{R}^3$
		$$a_1x_1 + a_2x_2 + a_3x_3 = b$$
	\item[$n>2$] hyperplane in $\mathbb{R}^{n+1}$
		$$a_1x_1 + a_2x_2 + \cdots + a_{n+1}x_{n+1} = b$$
\end{description}
\end{frame}

\begin{frame}{Parallel Planes}
\begin{definition}[Parallel Planes]
	Two $n$-planes are parallel if they are of the form, $f^{-1}(b_1)$ and $f^{-1}(b_2)$ where $f(x_1,x_2,\cdots,x_{n+1}) = a_1x_1 + a_2x_2 + \cdots +a_{n+1}x_{n+1}$ and $b_1,b_2 \in \mathbb{R}$.
	%\[ a_1x_1 + a_2x_2 + \cdots + a_{n+1}x_{n+1} = b_1 \]
	%\[ a_1x_1 + a_2x_2 + \cdots + a_{n+1}x_{n+1} = b_2 \]
\end{definition}
\begin{exampleblock}{Example}
	The planes $P_1$ and $P_2$ are parallel.
	\[ P_1 : x_1 + 2x_2 - 3x_3 = 1 \]
	\[ P_2 : x_1 + 2x_2 - 3x_3 = 2 \]
\end{exampleblock}
\end{frame}

\begin{frame}{Cylinder over a Surface}
\begin{definition}
Let $S$ be an $(n-1)$ surface in $\mathbb{R}^n$, given by $S = f^{-1}(c)$,\\
where $f : U \to \mathbb{R}$ such that $\nabla f(p) \ne 0,\ \forall p \in S$
	$$\nabla f(p) = \left(p,\frac{\partial f}{\partial x_1}(p), \frac{\partial f}{\partial x_2}(p),\cdots, \frac{\partial f}{\partial x_n}(p)\right) \ne (p,0),\ \forall p \in S$$

Let $g : U \times \mathbb{R} \to \mathbb{R}$, where $g(x_1,x_2,\cdots,x_{n+1}) = f(x_1,x_2,\cdots,x_n)$.
	$$\nabla g(q) = \left(q,\frac{\partial f}{\partial x_1}(q), \frac{\partial f}{\partial x_2}(q),\cdots, \frac{\partial f}{\partial x_n}(q),0 \right) \ne (q,0),\ \forall q \in g^{-1}(c)$$

This $n$-surface $g^{-1}(c)$ is the \textbf{cylinder over $S$}.
\end{definition}
\end{frame}

\begin{frame}{Cylinder over Surface : Example}
\begin{itemize}
	\item $S : x_1^2+x_2^2 = 1$, unit circle.
	\item $S = f^{-1}(1)$ where $f(x_1,x_2) = x_1^2 + x_2^2$.
	\item $\nabla f(x_1,x_2) = (x_1,x_2,2x_1,2x_2) \ne (x_1,x_2,0,0)$, since $(0,0) \notin S$
	\item<2> $g(x_1,x_2,x_3) = x_1^2 + x_2^2$
	\item<2> $g^{-1}(1)$ is the usual cylinder in $\mathbb{R}^3$ and
	\item<2> $\nabla g(x_1,x_2,x_3) = (x_1,x_2,x_3,2x_1,2x_2,0) \ne (x_1,x_2,x_3,0,0,0)$\\
		\hspace{2in} since $(0,0,z) \notin g^{-1}(1)$
\end{itemize}
\begin{center}
\tdplotsetmaincoords{70}{30}
\begin{tikzpicture}[tdplot_main_coords,scale=0.4]
	\draw[->] (0,-4,0) -- (0,4,0) node[above right] {$x_1$};
	\draw[->] (-4,0,0) -- (4,0,0) node[below right] {$x_2$};
	\draw<2>[->] (0,0,-4) -- (0,0,4) node[below right] {$x_3$};
	%surface
	\draw[red] plot[variable=\x,domain=0:360,samples=180] ({cos(\x)},{sin(\x)},0);
	%cylinder over surface
	\draw<2> plot[variable=\x,domain=0:360,samples=180] ({cos(\x)},{sin(\x)},-1.25);
	\draw<2> plot[variable=\x,domain=0:360,samples=180] ({cos(\x)},{sin(\x)},1.25);
	%traces of extension
	\foreach \x in {210,30}
		{\draw<2> ({cos(\x)},{sin(\x)},-1.25) -- ({cos(\x)},{sin(\x)},1.25);}
\end{tikzpicture}
\end{center}
\end{frame}

%IMPORTANT x-y axis swap
\begin{frame}{Cylinder over Surface : Examples}
\begin{figure}
\tdplotsetmaincoords{70}{30}
%line to cylinder
\begin{tikzpicture}[tdplot_main_coords,scale=0.2]
	\draw[->] (0,-10,0) -- (0,10,0) node[above right] {$x_1$};
	\draw[->] (-10,0,0) -- (10,0,0) node[below right] {$x_2$};
	%\draw[->] (0,0,-10) -- (0,0,10) node[below right] {$x_3$};
	%surface	
	\draw[red] (0,0,0) circle (1mm);
	%cylinder over surface
	\draw[blue] (-8,0,0) -- (8,0,0);
\end{tikzpicture}
%parabola to cylinder
\begin{tikzpicture}[tdplot_main_coords,scale=0.15]
	\draw[->] (0,-10,0) -- (0,10,0) node[above right] {$x_1$};
	\draw[->] (-10,0,0) -- (10,0,0) node[below right] {$x_2$};
	\draw[->] (0,0,-10) -- (0,0,10) node[below right] {$x_3$};
	%surface	
	\draw[red] plot[variable=\x,smooth,domain=-3:3,samples=180] (\x,\x*\x,0);
	%cylinder over surface
	\draw plot[variable=\x,domain=-3:3,samples=180] (\x,\x*\x,-3);
	\draw plot[variable=\x,domain=-3:3,samples=180] (\x,\x*\x,3);
	%boundary of extension
	\foreach \x in {-3,2.99}
		{\draw (\x,\x*\x,-3) -- (\x,\x*\x,3);}
\end{tikzpicture}
%ellipse to cylinder
\tdplotsetmaincoords{70}{30}
\begin{tikzpicture}[tdplot_main_coords,scale=0.2]
	\draw[->] (0,-6,0) -- (0,6,0) node[above right] {$x_1$};
	\draw[->] (-6,0,0) -- (6,0,0) node[below right] {$x_2$};
	\draw[->] (0,0,-6) -- (0,0,6) node[below right] {$x_3$};
	%surface testing	
	\draw[red] plot[variable=\x,smooth,domain=-pi:pi,samples=180] ({sin(deg(\x))*3},{cos(deg(\x))*2},0);
	%cylinder over surface
	\draw plot[variable=\x,domain=-pi:pi,samples=180] ({sin(deg(\x))*3},{cos(deg(\x))*2},-3);
	\draw plot[variable=\x,domain=-pi:pi,samples=180] ({sin(deg(\x))*3},{cos(deg(\x))*2},3);
	%boundary of extension
	\foreach \x in {1.4,4.5}
		{\draw ({sin(deg(\x))*3},{cos(deg(\x))*2},-3) -- ({sin(deg(\x))*3},{cos(deg(\x))*2},3);}
\end{tikzpicture}
\tdplotsetmaincoords{70}{30}
\begin{tikzpicture}[tdplot_main_coords,scale=0.15]
	\draw[->] (0,-8,0) -- (0,8,0) node[above right] {$x_1$};
	\draw[->] (-8,0,0) -- (8,0,0) node[below right] {$x_2$};
	\draw[->] (0,0,-8) -- (0,0,8) node[below right] {$x_3$};
	%surface	
	\draw[red] plot[variable=\x,smooth,domain=-pi:pi,samples=180] ({sin(deg(\x))},\x,0);
	%cylinder over surface
	\draw plot[variable=\x,domain=-pi:pi,samples=180] ({sin(deg(\x))},\x,-3);
	\draw plot[variable=\x,domain=-pi:pi,samples=180] ({sin(deg(\x))},\x,3);
	%traces of extension
	\foreach \x in {-pi,pi}
		{\draw ({sin(deg(\x))},\x,-3) -- ({sin(deg(\x))},\x,3);}
\end{tikzpicture}
\end{figure}
\end{frame}

\begin{frame}{Surface of Revolution}
\begin{block}{Obtained by rotating a curve about an axis}
\begin{itemize}
	\item $f : U \to \mathbb{R},\  \nabla f(p) \ne 0,\ \forall p \in U$
	\item $C = f^{-1}(c)$ where $U \subset \mathbb{R}^2$ with $x_2 > 0$ \\
		That is, $C$ is a curve in $\mathbb{R}^2$  not touching the $x_1$ axis
	\item<2> $g : U \times \mathbb{R} \to \mathbb{R},\ g(x_1,x_2,x_3) = f(x_1,(x_2^2+x_3^2)^{\frac{1}{2}})$
	\item<2> $S = g^{-1}(c)$ is a surface of revolution of $C$ about $x_1$ axis.
\end{itemize}
\end{block}
\begin{figure}
\tdplotsetmaincoords{70}{30}
\begin{tikzpicture}[tdplot_main_coords,scale=0.3]
	\draw[->] (0,-5,0) -- (0,5,0) node[above right] {$x_1$};
	\draw[->] (-5,0,0) -- (5,0,0) node[below right] {$x_2$}; %swapped
	\draw[->] (0,0,-5) -- (0,0,5) node[below right] {$x_3$};
	%curve	
	\draw[red] plot[variable=\x,smooth,domain=-4:4,samples=180] ({sin(deg(\x))+4},\x,0);
	%traces of revolution
	\foreach \y in {-4,-3.8,...,4}
		{\draw<2>[thin,dashed] plot[variable=\x,smooth,domain=-pi:pi,samples=180] ({(sin(deg(\y))+4)*cos(deg(\x))},\y,{(sin(deg(\y))+4)*sin(deg(\x))});}
\end{tikzpicture}
\end{figure}
\end{frame}

\begin{frame}{Surface of revolution of Curve : Examples}
\begin{figure}
%revolution of a line
\tdplotsetmaincoords{70}{30}
\begin{tikzpicture}[tdplot_main_coords,scale=0.5]
	\draw[->] (0,-3,0) -- (0,3,0) node[above right] {$x_1$};
	\draw[->] (-3,0,0) -- (3,0,0) node[below right] {$x_2$}; %swap
	\draw[->] (0,0,-3) -- (0,0,3) node[below right] {$x_3$};
	%curve	
	\draw[red] plot[variable=\x,smooth,domain=-2:2,samples=180] (1,\x,0);
	%surface of revolution
	\draw plot[variable=\x,smooth,domain=-2:2,samples=180] ({cos(deg(2))},\x,{sin(deg(2))});
	\draw plot[variable=\x,smooth,domain=-2:2,samples=180] ({cos(deg(5.2))},\x,{sin(deg(5.2))});

	\draw plot[variable=\x,smooth,domain=-pi:pi,samples=180] ({cos(deg(\x))},2,{sin(deg(\x))});
	\draw plot[variable=\x,smooth,domain=-pi:pi,samples=180] ({cos(deg(\x))},-2,{sin(deg(\x))});
	%traces of revolution 
	\foreach \y in {-2,-1.6,...,2}
		{\draw[thin,dashed] plot[variable=\x,smooth,domain=-pi:pi,samples=180] ({cos(deg(\x))},\y,{sin(deg(\x))});}
\end{tikzpicture}
%revolution into hyperboloid
\tdplotsetmaincoords{70}{30}
\begin{tikzpicture}[tdplot_main_coords,scale=0.15]
	\draw[->] (0,-10,0) -- (0,10,0) node[above right] {$x_1$};
	\draw[->] (-10,0,0) -- (10,0,0) node[below right] {$x_2$}; %swap
	\draw[->] (0,0,-10) -- (0,0,10) node[below right] {$x_3$};
	%curve	
	\draw[red] plot[variable=\x,smooth,domain=-8:8,samples=180] ({sqrt(1+\x*\x)},\x,0);
	\draw[red,dashed] plot[variable=\x,smooth,domain=-8:8,samples=180] ({-sqrt(1+\x*\x)},\x,0);
	%surface of revolution
	\draw[dashed] plot[variable=\x,smooth,domain=-7:7,samples=180] ({sqrt(1+\x*\x)*cos(deg(0.8))},\x,{sqrt(1+\x*\x)*sin(deg(0.8))});
	\draw[dashed] plot[variable=\x,smooth,domain=-7:7,samples=180] ({sqrt(1+\x*\x)*cos(deg(5))},\x,{sqrt(1+\x*\x)*sin(deg(5))});
	\draw[dashed] plot[variable=\x,smooth,domain=-7:7,samples=180] ({sqrt(1+\x*\x)*cos(deg(1.5))},\x,{sqrt(1+\x*\x)*sin(deg(1.5))});
	\draw[dashed] plot[variable=\x,smooth,domain=-7:7,samples=180] ({sqrt(1+\x*\x)*cos(deg(4))},\x,{sqrt(1+\x*\x)*sin(deg(4))});

	\draw plot[variable=\x,smooth,domain=-pi:pi,samples=180] ({sqrt(50)*cos(deg(\x))},7,{sqrt(50)*sin(deg(\x))});
	\draw plot[variable=\x,smooth,domain=-pi:pi,samples=180] ({sqrt(50)*cos(deg(\x))},-7,{sqrt(50)*sin(deg(\x))});%50 = 1+7^2
	%traces of revolution
	\foreach \y in {-7,-6.3,...,7}
		{\draw[dashed] plot[variable=\x,smooth,domain=-pi:pi,samples=180] ({sqrt(1+\y*\y)*cos(deg(\x))},\y,{sqrt(1+\y*\y)*sin(deg(\x))});}
\end{tikzpicture}
\tdplotsetmaincoords{70}{30}
%revolution into torus
\begin{tikzpicture}[tdplot_main_coords,scale=0.5]
	\draw[->] (0,-4,0) -- (0,4,0) node[above right] {$x_1$};
	\draw[->] (-4,0,0) -- (4,0,0) node[below right] {$x_2$}; %swap
	\draw[->] (0,0,-4) -- (0,0,4) node[below right] {$x_3$};
	%curve	
	\draw[red] plot[variable=\x,smooth,domain=-pi:pi,samples=180] ({2+sin(deg(\x))},{cos(deg(\x))},0);
	%surface of revolution
	%\draw[dashed] plot[variable=\x,smooth,domain=-1:1,samples=180] ({(sqrt(1-\x*\x)+2)*cos(deg(3))},\x,{(sqrt(1-\x*\x)+2)*sin(deg(3))});
	%\draw[dashed] plot[variable=\x,smooth,domain=-1:1,samples=180] ({(sqrt(1-\x*\x)+2)*cos(deg(-3))},\x,{(sqrt(1-\x*\x)+2)*sin(deg(-3))});
	%traces of revolution
	\foreach \y in {-180,-170,...,180}
		{\draw[thin,dashed] plot[variable=\x,smooth,domain=-1:1,samples=180] ({(sqrt(1-\x*\x)+2)*cos(\y)},\x,{(sqrt(1-\x*\x)+2)*sin(\y)});
		\draw[thin,dashed] plot[variable=\x,smooth,domain=-1:1,samples=180] ({(-sqrt(1-\x*\x)+2)*cos(\y)},\x,{(-sqrt(1-\x*\x)+2)*sin(\y)});}
\end{tikzpicture}
\end{figure}
\end{frame}

\begin{frame}{Extreme Points}
\begin{definition}[Extreme Point]
	Let an $n$-surface $S \subset U$ and $g : U \to \mathbb{R}$ be a smooth function.
	Then $p \in S$ is an extreme point of $g$ on the surface $S$ if 
	\begin{itemize}
		\item $g(p) \le g(q),\ \forall q \in S$ or 
		\item $g(p) \ge g(q),\ \forall q \in S$.
	\end{itemize}
\end{definition}
\begin{theorem}
Let $S$ be an $n$-surface in $\mathbb{R}^{n+1}$, $S = f^{-1}(c)$ where $f : U \to \mathbb{R}$ such that $\nabla f(q) \ne 0,\ \forall q \in S$.
Let $g : U \to \mathbb{R}$ be a smooth function and $p \in S$ be an extreme point of $g$ on $S$. Then there exists a real number $\lambda$ such that $\nabla g(p) = \lambda \nabla f(p)$.
\end{theorem}
\end{frame}

\begin{frame}{Extreme Points : Proof}
\begin{itemize}
	\item $v \in S_p \implies v = \dot{\alpha}(t_0),\ \alpha : I \to S,\ \alpha(t_0) = p$
	\item If $p$ is an extreme point of $g$,\\
		 then $t_0$ is an extreme point of $g \circ \alpha$
	\begin{align*}
		(g \circ \alpha)'(t_0) & = 0  \\
       		\nabla g(\alpha(t_0)) \cdot \dot{\alpha}(t_0) & = 0 \\
		\nabla g(p)\cdot v & = 0
	\end{align*}
	\item $\nabla g(p) \in S_p^\perp$
	\item $[\nabla f(p)]^\perp = S_p$
	\item If $g(p) \in S_p^\perp$, then $\nabla g(p) = \lambda \nabla f(p)$\\
		Since, $S_p^\perp$ is 1-dimensional and is spanned by $\nabla f(p)$
\end{itemize}
\end{frame}

%\begin{frame}{Application : Extreme Points}
%\begin{itemize}
%	\item Let $g$ be a smooth function on an $n$-surface $S$ in $\mathbb{R}^{n+1}$.
%	\item $\nabla g(p) = \lambda \nabla f(p) \not\!\!\implies p \in S$ is an extreme point.
%	\item $\nabla g(p) \ne \lambda \nabla f(p), \forall \lambda \in \mathbb{R} \implies p \in S$ is not an extreme point.
%\end{itemize}
%\end{frame}

\end{document}
