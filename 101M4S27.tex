\documentclass{beamer}
\usepackage{pgf,tikz}

\newcommand{\subgroup}{\le}
\newcommand{\subring}{\le}
\newcommand{\subfield}{\le}
\newcommand{\isomorphism}{\simeq}

%Text Books : \cite{fraleigh}
%Module 1
%Direct products and finitely generated Abelian groups, fundamental theorem, Applications
%Factor groups, Fundamental homomorphism theorem, normal subgroups and inner automorphisms.
%Group action on a set, Isotropy subgroups, Applications of G- sets to counting.
%(Part II – Sections 11, 14, 16 & 17) (25 hours)
%Module 2
%Isomorphism theorems, Sylow theorems , Applications of the Sylow theory.
%(Part VII Sections 34, 36 & 37) (25 hours)
%Module 3
%Fermat’s and Euler Theorems, The field of quotients of an integral domain,
%Rings of polynomials, Factorisation of polynomials over a field.
%(Part IV – Sections 20, 21, 22 & 23) (20 hours)
%Module 4
%Non commutative examples, Homeomorphisms and factor rings, Prime and
%Maximal Ideals
%(Part V – Sections 24, 26 & 27) (20 hourStep 3 : $\psi$ is onto.
\title{Abstract Algebra}
\author{Module 4}

\institute{Section 27 : Prime and Maximal Ideals}

\begin{document}
\begin{frame}
	\maketitle
\end{frame}

\begin{frame}{Ring vs Factor Ring}
\begin{itemize}
	\item Ring may have \textbf{stronger} as well as \textbf{weaker} algebraic structure compared to its Factor Ring.
	\begin{itemize}
		\item $\mathbb{Z}/p\mathbb{Z}$ is a Field. But, $\mathbb{Z}$ is not a Field.
		\item $\mathbb{Z}/6\mathbb{Z}$ is not an Integral Domain. But, $\mathbb{Z}$ is an Integral Domain.
	\end{itemize}
	\item Every ring $R$ has two ideals,
	\begin{itemize}
		\item Improper Ideal $R$
		\item Trivial Ideal $\{0\}$
	\end{itemize}
\end{itemize}

\begin{definition}[Ideal]
	A subgroup of a ring $R$ is an ideal if $rN \subset N\ \&\ Nr \subset N,\ \forall r \in R$.
\end{definition}

\begin{definition}[Unit]
	Unit is an element which has multiplicative inverse.
\end{definition}

\end{frame}

\begin{frame}{Ideal with Unit}
\begin{theorem}[Ideal with Unit]
	Let $N$ be an ideal of ring $R$. Let $u$ be a unit in $N$.
	Then $N = R$.
\end{theorem}
\begin{proof}
	$N \underset{ideal}{\subgroup} R \implies \forall r \in R,\ rN \subset N$\\
	Unit, $u \in N \implies \exists u^{-1} \in R,\ u^{-1}u = 1 \in N$\\
	$r \in R,\ 1 \in N \implies r1 = r \in N$
\end{proof}
\begin{corollary}
	A field contains no proper, nontrivial ideals.
\end{corollary}
\end{frame}

\begin{frame}{Maximal Ideal, Prime Ideal}
\begin{definition}[Maximal Ideal]
	A proper ideal which is not contained in any other proper ideal.
\end{definition}
\begin{exampleblock}{$p\mathbb{Z} \subset \mathbb{Z}$}
	Let $p$ be a prime. Then $p\mathbb{Z}$ is a maximal ideal of $\mathbb{Z}$ (Why ?)
\end{exampleblock}
\begin{definition}[Prime Ideal]
	A proper  ideal $N$ is prime ideal if $ab \in N \implies a \in N \text{ or } b \in N$.
\end{definition}
\begin{exampleblock}{$\{ 0,2 \} \subset \mathbb{Z}_4$}
$\{ 0,2 \}$ is a prime ideal of $\mathbb{Z}_4$ \\
$ 0 = 0 \cdot x = x \cdot 0 = 2 \cdot 2$\\
$ 2 = 1 \cdot 2 = 2 \cdot 1$ \\
Remember that $1 = 1 \cdot 1 = 3 \cdot 3$\\
Thus, prime ideal of $\mathbb{Z}_4$ containing $1$ should also contain $3$.
\end{exampleblock}
\end{frame}

\begin{frame}{Maximal Ideal characterisation of Field}
\begin{theorem}
	Let $R$ be a commutative Ring with unity.\\
	$M$ is a maximal ideal of $R$ $\iff$ factor ring $R/M$ is a field
\end{theorem}
\begin{block}{Sufficient Part : Context}
\begin{itemize}
	\item Commutative Ring with unity $R$
	\item Maximal ideal $M$
	\begin{itemize}
		\item Ideal $\implies \forall r \in R, rM \subset M,\ Mr \subset M$
		\item Maximal $\implies M \subsetneq R$
	\end{itemize}
	\item $M$ is an ideal of $R \implies R/M$ is commutative ring with unity.
	\item $M$ is maximal  $\implies R/M$ is non-zero ie, $R/M \ne \{ 0+M \}$
	\item $R/M \ne \{ 0+M \} \implies {\color{red} \exists (a + M) \in R/M,\ (a+M) \ne (0+M)}$ That is, $a \notin M$ and  $a+M$ is not the additive identity of $R/M$
	\item $(a+M) \in R/M$ has multiplicative inverse $\implies R/M$ is a field
\end{itemize}
\end{block}
\end{frame}

\begin{frame}{Proof : Characterisation of Field}
\begin{block}{Sufficient Part}
\begin{itemize}
	\item Suppose $a+M$ doesn't have multiplicative inverse in $R/M$
	\item $(R/M)(a+M) = \{ (r+M)(a+M) : (r+M) \in R/M \}$
\begin{itemize}
	\item $x \in (R/M)(a+M) \implies x = (r+M)(a+M),\ r+M \in R/M$
\end{itemize}
	\item Claim : $(1+M) \notin (R/M)(a+M)$
\begin{itemize}
	\item Suppose $(1+M) \in (R/M)(a+M)$
	\item $\exists (b+M) \in R/M$ such that $(b+M)(a+M) = (1+M)$
	\item $ba = ab = 1$ is a contradiction.
\end{itemize}
	\item $(R/M)(a+M)$ is non-trivial, proper ideal of $R/M$
\begin{itemize}
	\item $(a+M) = (1+M)(a+M) \in (R/M)(a+M) \implies $ non-trivial
	\item $(1+M) \notin (R/M)(a+M) \implies$ proper
\end{itemize}
	\item  Canonical Homomorphism, $\gamma : R \to R/M,\ \gamma(a) = a+M$
\begin{itemize}
	\item $\ker(\gamma) = M \implies M \subset \gamma^{-1}[(R/M)(a+M)]$
	\item $\gamma^{-1}[(R/M)(a+M)]$ is a proper, ideal of $R$ containing $M$
\end{itemize}
	contradicts $M$ is maximal ideal in $R$
\end{itemize}
\end{block}
\end{frame}

\begin{frame}{Proof : Characterisation of Field}
\begin{block}{Necessary Part}
\begin{itemize}
	\item $M$ is ideal of $R$
	\item Suppose $R/M$ is a field.
	\item Suppose $M$ is not Maximal ideal of $R$\\
		$\exists \text{ proper, ideal }N$ containing $M$, ie, $N \subset M \subset R$
	\item Canonical Homomorphism, $\gamma : R \to R/M,\ \gamma(a) = a+M$
	\item $\gamma[N]$ is a proper, non-trivial ideal of $R/M$
	\begin{itemize}
		\item $\gamma[M] = \{ 0+M \} \subset \gamma[N] \implies $ non-trivial
		\item $N \ne R \implies \exists b \in R$ such that $b \notin N,\ \gamma(b) = (b+M) \notin \gamma[N] \implies$ proper
	\end{itemize}
	is a contradiction since $R/M$ is a field.\\
		\textbf{A field contains no proper, non-trivial ideal.}
\end{itemize}
\end{block}
\end{frame}

\begin{frame}{Ideal characterisation of Field}
\begin{corollary}
	A commutative ring with unity is a field if and only if it has no proper, non-trivial ideals.
\end{corollary}
\begin{block}{Sufficient Part}
\begin{itemize}
	\item Commutative ring unit unity, $R$
	\item Suppose $R$ is a field.
	\item $R$ has no proper, non-trivial ideal.
\end{itemize}
\end{block}
\begin{block}{Necessary Part}
\begin{itemize}
	\item Commutative ring unit unity, $R$
	\item Suppose $R$ has no proper, non-trivial ideal.
	\item Maximal ideal $\{ 0 \}$
	\item $R/\{ 0 \} \simeq R $ is a field
\end{itemize}
\end{block}
\end{frame}

\begin{frame}{Prime Ideal characterisation of Integral Domain}
\begin{theorem}
\begin{itemize}
	\item Commutative Ring $R$ with unity
	\item Proper ideal of $R$, $N \ne R$
	\item $N$ is prime ideal $\iff$ factor ring $R/N$ is integral domain
\end{itemize}
\end{theorem}
\begin{proof}
	$$(a+N)(b+N) = ab + N = 0 + N \iff ab \in N$$
\begin{itemize}
	\item $R/N$ Prime Ideal,  
		$ab \in N \implies a \in N \text{ OR } b \in N$
	$$a \in N \text{ OR } b \in N \iff a + N = N \text{ OR } b+N = N$$
	\item Integral Domain (No zero Divisors), 
		$(a+N)(b+N) = 0+N \implies (a+N) = N \text{ OR } (b+N) = N$
\end{itemize}
\end{proof}
\end{frame}

\begin{frame}{Corollary}
\begin{corollary}
	Every maximal ideal in a commutative ring $R$ with unity is a prime ideal.
\end{corollary}
\begin{proof}
	Suppose $M$ is an ideal of a commutative ring $R$ with unity
\begin{itemize}
	\item $M$ is maximal ideal of $R$
	\item $\implies R/M$ is a field
	\item $\implies R/M$ is an integral domain
	\item $\implies M$ is a prime ideal
\end{itemize}
\end{proof}
\end{frame}

\begin{frame}{Prime Field}
\begin{block}{Results}
\begin{enumerate}
	\item For ring $R$ with unity 1, the function $\phi : \mathbb{Z} \to R$\\
		\hspace{1em} defined by $\phi(n) = n \cdot 1$ is a ring homomorphism.
	\item For ring $R$ with characteristic $n > 1$,\\
		\hspace{5em} $R$ contains a subring isomorphic to  $\mathbb{Z}_n$
	\item For ring $R$ with characteristic $0$,\\
		\hspace{5em} $R$ contains a subring isomorphic to $\mathbb{Z}$
	\item For field $F$ with prime characteristic $p$,\\
		\hspace{5em} $F$ contains a subfield isomorphic to  $\mathbb{Z}_p$
	\item For field $F$ with characteristic $0$,\\
		\hspace{5em} $F$ contains a subfield isomorphic to $\mathbb{Q}$
\end{enumerate}
\end{block}
\begin{definition}[Prime Field]
	The fields $\mathbb{Z}_p$ and $\mathbb{Q}$ are prime fields. ({\color{blue} Why ?} Semester 2)
\end{definition}
\end{frame}

\begin{frame}{Proof : Ring Homomorphism given by $\phi : \mathbb{Z} \to R$}
\begin{proof}
	Let $1$ be the unity of the Ring $R$ with Unity
	\begin{align*}
		\phi(n+m) & = (n+m) \cdot 1 
		 = \underbrace{(1+1+\cdots+1)}_{\text{$n+m$ summands}} \\
		& = \underbrace{(1+1+\cdots+1)}_{\text{$n$ summands}} + \underbrace{(1+1+\cdots+1)}_{\text{$m$ summands}} \\
		& = (n \cdot 1)+(m \cdot 1) = \phi(n) + \phi(m)\\ \hline
		\phi(n)\phi(m) & = (n \cdot 1)(m \cdot 1) \\
		& = \underbrace{(1+1+\cdots+1)}_{\text{$n$ summands}} \underbrace{(1+1+\cdots+1)}_{\text{$m$ summands}} \\
		& = \underbrace{(1+1+\cdots+1)}_{\text{$nm$ summands}} \\
		& = (nm) \cdot 1 = \phi(nm)
	\end{align*}
\end{proof}
\end{frame}

\begin{frame}{Proof: $\mathbb{Z}_n$ subring of Ring of Characteristic $n > 1$}
\begin{itemize}
	\item Suppose $R$ is
	\begin{itemize}
		\item a commutative ring with unity $1$ and 
		\item characteristic $n$, ($n > 1$)
	\end{itemize}
	\item $\phi : \mathbb{Z} \to R,\ \phi(m) = m \cdot 1$ is a (ring) Homomorphism
	\begin{itemize}
		\item $\phi(1) = 1.1 = (1)$
		\item $\phi(2) = 2 \cdot 1 = (1+1) $ \\
		$\vdots$
		\item $\phi(n-1) = (n-1) \cdot 1 = \underbrace{(1+1+\cdots+1)}_{\text{$n-1$ summands}}$
		\item $\phi(n) = n \cdot 1 = \underbrace{(1+1+\cdots+1)}_{\text{$n$ summands}} = 0 {\color{red} = \phi(0)}$
	\end{itemize}
	\item {\color{blue}Kernel $\ker(\phi) = \{ \cdots,-2n,-n,0,n,2n,\cdots\} = n\mathbb{Z}$}
	\item {\color{blue}For any $n>1$, $n\mathbb{Z}$ is an ideal of $\mathbb{Z}$.}
	\begin{itemize}
		\item $\phi : R' \to R$ is ring homomorphism $\iff \ker(\phi)$ is an ideal of $R$
	\end{itemize}
\item $\mathbb{Z}_n \underset{\text{congruence}}{\simeq} \mathbb{Z}/n\mathbb{Z} = \mathbb{Z}/\ker(\phi) \underset{\text{canonical}}{\simeq} \phi[\mathbb{Z}] \le R$
\end{itemize}
\end{frame}
\begin{frame}{Proof: $\mathbb{Z}$ subring of Ring of Characteristic $0$ }
\begin{itemize}
	\item $R$ is a commutative ring with unity $1$ and characteristic $0$
	\item $\phi : \mathbb{Z} \to R,\ \phi(n) = n \cdot 1$ is a ring homomorphism
	\item Claim : $\ker(\phi) = \{ 0 \}$\\
		Suppose kernel is nontrivial, there exists $m \in \ker(\phi),\ m \ne 0$\\
		Then $m \in \ker(\phi) \implies \phi(m) = 0 \implies \text{ characteristic } \ne 0 $
	\item $\mathbb{Z} \underset{\text{trivial}}{\simeq} \mathbb{Z}/\{0\} = \mathbb{Z}/\ker(\phi) \underset{\text{canonical}}{\simeq} \phi[\mathbb{Z}] \le R$
\end{itemize}
\end{frame}

\begin{frame}{Proof: $\mathbb{Z}_p$ subfield of Field of Prime Characteristic $p$}
\begin{itemize}
	\item Field $F$ with characteristic $n$, ($n > 1$)
	\item $\phi : \mathbb{Z} \to F$ is a homomorphism.
	\item $\mathbb{Z}_n$ is a subring of $F$, $\mathbb{Z}_n \underset{\text{ring}}{\le} F$
	\item Claim : $n$ is a prime.
	\begin{itemize}
		\item If $n$ is not a prime, then $n = ab, a>1,b>1$
		\item $\implies \mathbb{Z}_n$ has zero divisors $a,b$
		\item $\implies F$ has zero divisors $\phi(a),\phi(b)$
	\end{itemize}
	\item $\langle \mathbb{Z}_p,+_p,\times_p \rangle$ is a field
	\item $\mathbb{Z}_p$ is a subfield of $F$, $\mathbb{Z}_p \underset{\text{field}}{\le} F$
\end{itemize}
\end{frame}

\begin{frame}{Proof: $\mathbb{Q}$ subfield of Field of Characteristic $0$}
\begin{itemize}
	\item Field $F$ with characteristic $0$
	\item $\mathbb{Z} \underset{\text{ring}}{\le} F$
	\item $\mathbb{Z}$ is an integral domain
	\item $\mathbb{Q}$ is the field of quotients of $\mathbb{Z}$ (refer : Fraleigh \S21)\\
		The smallest field containing the integral domain
	\item $\mathbb{Q} \underset{\text{field}}{\le} F$
\end{itemize}
\begin{block}{Field of characteristic $1$ ?}
	Field of characteristics $1$ does not exists.\\
	Characterstic $1 \implies 1 = 0$ is contradictory as $F = \{ 0 \}$\\
	Thus smallest/trivial field is $Z_2 = \{ 0 ,1 \}$
\end{block}
\end{frame}

\begin{frame}{Principal ideal}
\begin{definition}[ideal generated by an element]
	Let $R$ be a commutative ring with unity and $a \in R$. The ideal generated by $a$ is the set of all elements of the form $ra$ where $r \in R$.
	$$ \langle a \rangle = \{ ra : r \in R \}$$
\end{definition}
\begin{center}\textbf{It is the smallest ideal containing $a$.}\end{center}
\begin{definition}[Principal ideal]
	An ideal with a generator
\end{definition}
\begin{exampleblock}{Ideals of $\mathbb{Z}$}
	Every ideal of $\mathbb{Z}$ is a principal ideal.\\
	An ideal of $\mathbb{Z}$ is of the form $n\mathbb{Z} = \langle n \rangle$
\end{exampleblock}
\begin{exampleblock}{Ideal $\langle x \rangle$ of $F[x]$}
	$\langle x \rangle$ in $F[x]$ is the set of all polynomials with zero constant term.
\end{exampleblock}
\end{frame}

\begin{frame}{Polynomials over field $F$, $F[x]$}
\begin{theorem}
	Every ideal in $F[x]$ is a principal ideal.
\end{theorem}
\begin{proof}
\begin{itemize}
	\item $N$ be an ideal of $F[x]$, $N = \{ 0 \} \implies N = \langle 0 \rangle$
	\item Suppose $N \ne \{ 0 \}$. There exists a polynomial of minimum degree $g(x) \in N,\ g(x) \ne 0$
	\item Case 1 : degree of $g(x) = 0$
	\begin{itemize}
		\item $g(x)$ is a constant. $g(x) \in F$. And has multiplicative inverse.
		\item ideal $N$ contains unit $g(x) \implies N = F[x] = \langle 1 \rangle$
	\end{itemize}
	\item Case 2 : degree of $g(x) \ge 1$
	\begin{itemize}
		\item  $f(x) \in N \implies f(x) = q(x)g(x) + r(x)$ where degree of $r(x)$ is strictly less than the degree of $g(x)$\\
		\item $r(x) \ne 0$ is a contradiction since degree of $g(x)$ is not minimum in $F[x]$ as $r(x) = f(x)-q(x)g(x) \in F[x]$
		\item $f(x) \in N \implies f(x) = q(x)g(x) \implies F[x] = \langle g(x) \rangle$ 
	\end{itemize}
\end{itemize}
\end{proof}
\end{frame}


\begin{frame}{Ideal generated by Irreducible Polynomials}
	$$\langle p(x) \rangle = \{ 0 \} \iff p(x) = 0 \in F[x]$$
\begin{theorem}
	Non-trivial ideal $\langle p(x) \rangle$ is maximal $\iff p(x)$ is irreducible over $F$
\end{theorem}
\begin{block}{Proof : Sufficient Part}
\begin{itemize}
	\item Suppose $\langle p(x) \rangle$ is maximal ideal in $F[x]$
	\item Claim : $p(x) \notin F$
	\begin{itemize}
		\item $p(x) \in F \implies p(x)$ is a unit $\implies \langle p(x) \rangle = F[x]$\\
		$\implies p(x)$ is not a proper ideal\\
		$\implies \langle p(x) \rangle$ is not a maximal ideal
	\end{itemize}
	\item Suppose $p(x)$ is reducible, $p(x) = f(x)g(x)$
	\begin{itemize}
		\item Every maximal ideal is also prime ideal \\
		$f(x)g(x) \in \langle p(x) \rangle \implies f(x) \in \langle p(x) \rangle$ OR $g(x) \in \langle p(x) \rangle$
		\item $degree(f(x)) < degree(p(x)) \implies f(x) \notin \langle p(x) \rangle$
	\end{itemize}
	\item By contradiction, $p(x)$ is irreducible.
\end{itemize}
\end{block}
\end{frame}

\begin{frame}{ideal generated by irreducible polynomial}
\begin{block}{Proof : Necessary Part}
\begin{itemize}
	\item Suppose $p(x)$ is irreducible over $F$
	\item Suppose $\langle p(x) \rangle$ is not maximal
	\begin{itemize}
		\item There exists proper ideal $N$ properly containing $\langle p(x) \rangle$\\ That is, $\exists \text{ ideal } N \text{ such that } \langle p(x) \rangle \subsetneq N \subsetneq F[x]$
	\end{itemize}
	\item Every ideal in $F[x]$ is principal
	\begin{itemize}
		\item $N = \langle g(x) \rangle$ for some $g(x) \in F[x]$
		\item $p(x) \in N \implies p(x) = q(x)g(x)$
		\item $p(x)$ irreducible $\implies q(x)$ is of degree $0 \implies N = \langle p(x) \rangle$\\
			since $degree(g(x)) = 0 \implies g(x) \text{ is unit } \implies \langle g(x) \rangle = F[x]$
	\end{itemize}
	\item By contradiction, there is no proper ideal $N$ containing $\langle p(x) \rangle$
	\begin{itemize}
		\item $p(x)$ is irreducible $\implies degree(p(x)) \ge 1$\\
		$\implies \langle p(x) \rangle \ne \langle 1 \rangle = F[x]$
		That is, $\langle p(x) \rangle $ is a proper ideal \\
		$\implies \langle p(x) \rangle $ is a maximal ideal
	\end{itemize}
\end{itemize}
\end{block}
\end{frame}

\begin{frame}{Unique Factorisation in $F[x]$}
\begin{theorem}
\begin{itemize}
	\item $p(x)$ an irreducible polynomial in $F[x]$.
	\item $p(x) | r(x)s(x),\ r(x),s(x) \in F[x] \implies p(x) | r(x) \text{ OR } p(x) | s(x)$\\
\end{itemize}
\end{theorem}
\begin{proof}
\begin{itemize}
	\item $p(x) | r(x)s(x) \implies r(x)s(x) \in \langle p(x) \rangle$ 
	\item $\langle p(x) \rangle$ is a prime field
	\item $r(x) \in \langle p(x) \rangle$ OR $s(x) \in \langle p(x) \rangle$
	\item WLOG $r(x) \in \langle p(x) \implies p(x) | r(x)$
\end{itemize}
\end{proof}
\begin{theorem}[Unique Factorisation]
	Every polynomial $p(x) \in F[x]$ has unique factorisation except for order and unit.
\end{theorem}
\end{frame}

\begin{frame}
	\vspace{0.6in}
	\hspace{3cm} {\color{blue}\Huge{Thank You}}
\end{frame}
\end{document}
