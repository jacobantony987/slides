\documentclass{beamer}

\newcommand{\isomorphism}{\simeq}

\title{Graph Theory}
\author{Module 1}
\institute{Section 4 : Degree of Vertices}

\begin{document}
\begin{frame}
	\maketitle
\end{frame}

\begin{frame}
\frametitle{Degree of a Vertex}
\begin{definition}[Degree]
	The degree of a vertex $v$ of $G$, $\deg_G(v)$ is the number of edge incident on it except for the loops at $v$, which are counted twice.
\end{definition}

\begin{description}
	\item[$\delta(G)$] minimum of the degrees of vertices of graph $G$.
	\item[$\Delta(G)$] maximum of the degrees of vertices of graph $G$.
\end{description}
\end{frame}

\begin{frame}
\frametitle{Regular Graph}
\begin{definition}[k-regular]
	A graph $G$ is $k$-regular for some non-negative integer $k$ if every vertex of $G$ has degree $k$.
\end{definition}

\begin{definition}[regular]
	A graph $G$ is regular if it is $k$-regular for some non-negative\\ integer $k$.
\end{definition}

\begin{definition}[cubic]
	A $3$-regular graph is called a cubic graph.
\end{definition}

\begin{definition}[1-factor]
	A spanning, 1-regular subgraph of graph $G$ is called a $1$ factor or\\ a perfect matching of $G$.
\end{definition}
\end{frame}

\begin{frame}
\frametitle{Degree Sequence}
\begin{description}
	\item[isolated vertex] is a vertex with degree $0$.
	\item[pendent vertex] is a vertex with degree $1$.
	\item[pendent edge] is an edge incident with a pendent vertex.
\end{description}
\begin{definition}[degree sequence]
	A finite sequence of degrees of vertices of $G$ which is either in nonincreasing or in nondecreasing order is called a degree sequence.
\end{definition}

\begin{definition}[graphical sequence]
	A finite sequence $d : d_1,d_2,\dots,d_n$ is graphical if there exists a simple graph with degree sequence $d$.
\end{definition}
\end{frame}

\begin{frame}
\frametitle{First Theorem}
\begin{theorem}[Euler]
	The sum of degrees of  vertices of a graph is equal to twice the number of its edges.
	$$ \sum_{i = 1}^n d_i = 2m $$
\end{theorem}

\begin{corollary}
	The number of odd vertices is even.
\end{corollary}
\end{frame}

\begin{frame}
\frametitle{Exercises 1}
	If $G,H$ are isomorphic, then each pair of corresponding vertices of $G,H$ have same degree.
\end{frame}

\begin{frame}
\frametitle{Exercises 2}
	Let $d : d_1,d_2,\dots,d_n$ be a degree sequence of a graph $G$ and $r$ be a positive integer. Then $\sum_{i = 1}^n d_i^r$ is even.
\end{frame}

\begin{frame}
\frametitle{Exercises 3}
	For every sequence $d : d_1,d_2,\dots,d_n$ with $\sum d_i$ even, there exists a (not necessarily simple) graph with degree sequence $d$.
\end{frame}

\begin{frame}
\frametitle{Exercises 4}
	In a group of $n$ persons ($n \ge 2$), there are at least two with the same number of friends.
\end{frame}

\begin{frame}
\frametitle{Exercises 5}
	Every vertex of a graph $G$ has degree either $k$ or $k+1$, then the number of vertices of degree $k$ is $(k+1)n-2m$.
\end{frame}

\begin{frame}
	\vspace{0.6in}
	\hspace{3cm} {\color{blue}\Huge{Thank You}}
\end{frame}
\end{document}
