\documentclass{beamer}

\usepackage{amsmath,amsthm,amsfonts}
\usepackage{mathrsfs,mathtools,mathdots}
\usepackage{caption,subcaption}
\usepackage{pgf,tikz}
\usepackage{pgfplots}
\usepackage{tikz-3dplot}
\usetikzlibrary{shapes.geometric,arrows,arrows.meta,decorations.pathreplacing,bending}

\title{Differential Geometry}
\author{Module III}
\institute{Chapter 9 : Weingarten Map}

\begin{document}

\begin{frame}
\maketitle
\end{frame}

\begin{frame}{Directional Derivatives}
	Suppose,
	$$ f : \mathbb{R}^{n+1} \to \mathbb{R} $$
	$$ \boldsymbol{X}(p) = \left(p,X_1(p),X_2(p),\dots,X_{n+1}(p) \right) $$ %where $X_j(p) : \mathbb{R}^{n+1} \to \mathbb{R}$.
\begin{itemize}
	\item $\nabla_{\boldsymbol{v}} f$, the derivative of function $f$ with respect to $\boldsymbol{v}$
		$$ \nabla_{\boldsymbol{v}} f = \nabla f(p) \cdot \boldsymbol{v} \text{ where } \boldsymbol{v} = (p,v)$$
	\item $\nabla_{\boldsymbol{v}} \boldsymbol{X}$, the derivative vector field $\boldsymbol{X}$ with respect to $\boldsymbol{v}$.
		$$ \nabla_{\boldsymbol{v}} \boldsymbol{X} = \left(p,\nabla_{\boldsymbol{v}} X_1,\nabla_{\boldsymbol{v}} X_2,\dots,\nabla_{\boldsymbol{v}} X_{n+1} \right) $$
	\item $D_{\boldsymbol{v}} \boldsymbol{X}$, the covariant derivative of $\boldsymbol{X}$ with respect to $\boldsymbol{v}$.
		$$ D_{\boldsymbol{v}} \boldsymbol{X} = \nabla_{\boldsymbol{v}} \boldsymbol{X} - \left( \nabla_{\boldsymbol{v}} \boldsymbol{X} \cdot \boldsymbol{N} \right) \boldsymbol{N} $$
\end{itemize}
\end{frame}

\begin{frame}{Properties of $\nabla_{\boldsymbol{v}} f$}
	$\nabla_{\boldsymbol{v}} f$ is a linear map. 
	\begin{enumerate}
		\item $\nabla_{\boldsymbol{v} + \boldsymbol{w}} f = \nabla_{\boldsymbol{v}} f + \nabla_{\boldsymbol{w}} f$.
		\item $\nabla_{c\boldsymbol{v}} f = c \nabla_{\boldsymbol{v}} f$.
	\end{enumerate}
	\vfill
	We also have,
	\begin{enumerate}
		\item $\nabla_{\boldsymbol{v}} (f+g) = \nabla_{\boldsymbol{v}} f + \nabla_{\boldsymbol{v}} g$.
		\item $\nabla_{\boldsymbol{v}} (fg) = f (\nabla_{\boldsymbol{v}} g) + (\nabla_{\boldsymbol{v}} f)g$.
	\end{enumerate}
\end{frame}

\begin{frame}{Properties of $\nabla_{\boldsymbol{v}} \boldsymbol{X}$ and $D_{\boldsymbol{v}} \boldsymbol{X}$}
	Properties of derivative of $\boldsymbol{X}$ with respect to $\boldsymbol{v}$,
	\begin{enumerate}
		\item $\nabla_{\boldsymbol{v}} (\boldsymbol{X}+\boldsymbol{Y}) = \nabla_{\boldsymbol{v}} \boldsymbol{X} + \nabla_{\boldsymbol{v}} \boldsymbol{Y}$.
		\item $\nabla_{\boldsymbol{v}} (f\boldsymbol{X}) = f (\nabla_{\boldsymbol{v}} \boldsymbol{X} + (\nabla_{\boldsymbol{v}} f) \boldsymbol{X}$.
		\item $\nabla_{\boldsymbol{v}} (\boldsymbol{X} \cdot \boldsymbol{Y}) = \boldsymbol{X} \cdot \nabla_{\boldsymbol{v}} \boldsymbol{Y} + (\nabla_{\boldsymbol{v}} \boldsymbol{X}) \cdot \boldsymbol{Y}$.
	\end{enumerate}
	\vfill
	Properties of covariant derivative of $\boldsymbol{X}$ with respect to $\boldsymbol{v}$,
	\begin{enumerate}
		\item $D_{\boldsymbol{v}} (\boldsymbol{X}+\boldsymbol{Y}) = D_{\boldsymbol{v}} \boldsymbol{X} + D_{\boldsymbol{v}} \boldsymbol{Y}$.
		\item $D_{\boldsymbol{v}} (f\boldsymbol{X}) = f (D_{\boldsymbol{v}} \boldsymbol{X}) + (\textcolor{red}{\nabla}_{\boldsymbol{v}} f) \boldsymbol{X}$.
		\item $D_{\boldsymbol{v}} (\boldsymbol{X} \cdot \boldsymbol{Y}) = \boldsymbol{X} \cdot D_{\boldsymbol{v}} \boldsymbol{Y} + (D_{\boldsymbol{v}} \boldsymbol{X}) \cdot \boldsymbol{Y}$.
	\end{enumerate}
\end{frame}

\begin{frame}{Weingarten Map, $L_p$}
	Weingarten map of an $n$-surface $S$ at $p$,
	$$ L_p : S_p \to S_p,\ L_p(\boldsymbol{v}) = -\nabla_{\boldsymbol{v}} \boldsymbol{N} $$
	\vfill
	Practically, $-\nabla_{\boldsymbol{v}} \boldsymbol{N}(p) = -\nabla_{\boldsymbol{v}} \tilde{\boldsymbol{N}}(p)$.\\
	\begin{itemize}
		\item $n$-sphere of radius $r$, $L_p(\boldsymbol{v}) = \frac{1}{r} \boldsymbol{v}$.
		%\item Hyperplane $f(p) = a \cdot p$, $L_p(\boldsymbol{v}) = $
		%\item Circular Cylinder, $L_p(\boldsymbol{v}) = $
	\end{itemize}
	\vfill
	The shape operator, $L_p$
	\begin{itemize}
		\item Rate of change of the unit normal at each point on $S$.
		\item Measure of the turning of tangent space at each point on $S$.
		\item Describes the shape of the surface $S$.
	\end{itemize}
\end{frame}

\begin{frame}{Properties of Weingarten Map}
	\begin{enumerate}
		\item The normal component of acceleration at $p$ is the same for all parametrized curves with same velocity at $p$.
			$$\ddot{\boldsymbol{\alpha}}(t_0) \cdot \boldsymbol{N}(p) = L_p(\boldsymbol{v}) \cdot \boldsymbol{v}$$
		\item Weingarten Map is self adjoint.
			$$ L_p(\boldsymbol{v}) \cdot \boldsymbol{w} = \boldsymbol{v} \cdot L_p(\boldsymbol{w}) $$
	\end{enumerate}
\end{frame}

\begin{frame}
	\vspace{0.6in}
	\hspace{3cm} {\color{blue}\Huge{Thank You}}
\end{frame}
\end{document}
