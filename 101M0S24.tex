\documentclass{beamer}

\newcommand{\subgroup}{\le}
\newcommand{\subfield}{\le}
\newcommand{\isomorphism}{\simeq}

%Text Books : \cite{fraleigh}
%Module 1
%Direct products and finitely generated Abelian groups, fundamental theorem, Applications
%Factor groups, Fundamental homomorphism theorem, normal subgroups and inner automorphisms.
%Group action on a set, Isotropy subgroups, Applications of G- sets to counting.
%(Part II – Sections 11, 14, 16 & 17) (25 hours)
%Module 2
%Isomorphism theorems, Sylow theorems , Applications of the Sylow theory.
%(Part VII Sections 34, 36 & 37) (25 hours)
%Module 3
%Fermat’s and Euler Theorems, The field of quotients of an integral domain,
%Rings of polynomials, Factorisation of polynomials over a field.
%(Part IV – Sections 20, 21, 22 & 23) (20 hours)
%Module 4
%Non commutative examples, Homeomorphisms and factor rings, Prime and
%Maximal Ideals
%(Part V – Sections 24, 26 & 27) (20 hourStep 3 : $\psi$ is onto.
\title{Abstract Algebra}
\author{Module 4}
\institute{Section 24 : Noncommutative Examples}

\begin{document}
\begin{frame}
	\maketitle
\end{frame}

\begin{frame}{$End(A)$ : Noncommutative Ring}
	\alert{$\langle End(A),+,\circ \rangle$ is well-defined only if $A$ is an abelian group.} 

	Let $\theta, \phi, \psi \in End(A),\ x \in A$
\begin{enumerate}
	\item<+-> Addition is associative. $\theta + (\phi + \psi) = (\theta + \phi ) + \psi$
	\item<+-> Addition is commutative. $\theta + \phi = \phi + \theta$
	\item<+-> Existence of Additive Identity. $\forall \theta \in End(A),\ \exists 0 \in End(A),\ \theta + 0 = \theta$
	\item<+-> Existence of Additive Inverses. $\forall \theta \in End(A),\ \exists \theta' \in End(A),\ \theta + \theta' = 0$
	\item<+-> Multiplication is associative. $\theta (\phi\psi)=(\theta\phi)\psi$
	\item<+-> Multiplication is distributive over Addition.
	\begin{enumerate}
		\item<.-> Multiplication is left-distributive over Addition.
			$\theta(\phi+\psi) = \theta\phi + \theta\psi$
		\item<.-> Multiplication is right-distributive over Addition.
			$(\phi+\psi)\theta = \phi\theta + \psi\theta$
	\end{enumerate}
	\item<+-> Existence of Multiplicative Identity. $\forall \theta \in End(A),\ \exists I \in End(A),\ \theta I = \theta = I\theta$
	\item<+-> Multiplication is non-commutative. $\theta\phi \ne \phi\theta$
\end{enumerate}
\end{frame}

\begin{frame}{Functions - Graduate Level Notions}
\begin{block}{well-defined function}
	Let $f : A \to B$ be a function. Then,
	\begin{enumerate}
		\item {\color<2>{red}$\forall x \in A,\ f(x) \in B$}
		\item {\color<3>{red}If $x_1 \xrightarrow{f} y_1$, and $x_2 \xrightarrow{f} y_2$, then $x_1 = x_2 \implies y_1 = y_2$}
	\end{enumerate}
\end{block}
\begin{exampleblock}{Example}
	\begin{itemize}
		\item {\color<4>{red} $f(x) = \sqrt{x}$}
			\only<4>{is not appreciated.}
		\item<4> $f : \mathbb{R}^+ \to \mathbb{R}^+,\ f(x) = \sqrt{x}$ is preferred.
		\item<2-> $f : \mathbb{R} \to \mathbb{R},\ f(x) = \sqrt{x}$ is not well-defined.
			Since, $f(-1) \notin \mathbb{R}$
		\item<3-> $f : \mathbb{R}^+ \to \mathbb{R},\ f(x) = \sqrt{x}$ is not well-defined.
			Since, $f(1) = ?$
		\item<4> $f : \mathbb{R}^+ \to \mathbb{R}^+,\ f(x) = \sqrt{x}$ is well-defined.
	\end{itemize}
\end{exampleblock}
\end{frame}

\begin{frame}{Algebraic Structure is well-defined ?}
\begin{block}{well-defined ring $\langle R,+,\ast \rangle$}
	\begin{enumerate}
		\item<+-> $R$ is a set. --- \textbf{usually true}
		\item<+-> The addition operation $+ : R \times R \to R$ is well-defined.
		\begin{enumerate}
			\item<+-> $\forall x,y \in R,\ +(x,y) = x + y \in R$ --- \textbf{closure property}
			\item<+-> {\color<+->{red}$(x_1,y_1) \xrightarrow{+} z_1$, $(x_2,y_2) \xrightarrow{+} z_2$. OR $x_1 + y_1 = z_1$, $x_2 + y_2 = z_2$\\
				If $x_1 = x_2$, $y_1 = y_2$, then $z_1 = z_2$} --- \textbf{well-defined} 
		\end{enumerate}
		\item<+-> The multiplication operation $\ast : R \times R \to R$ is well-defined.
		\begin{enumerate}
			\item<+-> $\forall x,y \in R,\ \ast(x,y) = x \ast y \in R$ --- \textbf{closure property}
			\item<+-> {\color<+->{red}$(x_1,y_1) \xrightarrow{\ast} z_1$, $(x_2,y_2) \xrightarrow{\ast} z_2$. OR $x_1 \ast y_1 = z_1$, $x_2 \ast y_2 = z_2$\\
			If $x_1 = x_2$, $y_1 = y_2$, then $z_1 = z_2$} --- \textbf{well-defined} 
		\end{enumerate}
	\end{enumerate}
	\begin{itemize}
		\item<+-> $\langle \mathbb{R},+,\star \rangle$, $x \star y = \sqrt{xy}$ is not well-defined.
		\begin{itemize}
			\item<+-> Ring multiplication is not closed, $-1 \star 1 \notin \mathbb{R}$\\
			\item<+-> Ring multiplication is not well-defined, $1 \star 1 = ?$
		\end{itemize}
		\item<+-> $\langle \mathbb{R}^+,+,\star \rangle$, $x \star y = \sqrt{xy}$ is well-defined. 
	\end{itemize}
\end{block}
\end{frame}

\begin{frame}{$\langle End(A),+,\circ \rangle$ is well-defined ?}
	Given $\langle A, \ast \rangle$ is an abelian group with idenitty $e$.
\begin{enumerate}
	\item $End(A)$ is a non-empty set, since $0 \in End(A)$\\
		where $0 : A \to A$ is defined by the relation $0(x) = e$.
	\item Addition is well-defined.
	\begin{enumerate}
		\item $\phi+\psi$ is an endomorphism of $A$ 
		\item $(\phi+\psi)(x)$ is uniquely defined for every $x \in A$ 
	\end{enumerate}
	\item Multiplication/composition is well-defined.
	\begin{enumerate}
		\item $\phi\psi$ is an endomorphism of $A$ 
		\item $(\phi\psi)(x)$ is uniquely defined for every $x \in A$.
	\end{enumerate}
\end{enumerate}
	Note : An endomorphism of $A$ is a homomorphism from $A$ into $A$.
\end{frame}

\begin{frame}{Step 1 : $\phi + \psi \in End(A)$}
	Let $\phi,\psi \in End(A)$, and $x \in A$.
	\begin{enumerate}
		\item function $\phi+\psi :A \to A$ is well-defined,
			$(\phi+\psi)(x) = \phi(x)+\psi(x) \in A$
			\begin{enumerate}
				\item $\phi,\psi$ are well-defined. $\phi(x),\psi(x) \in A$
				\item Group addition is closed.
			\end{enumerate}
		\item $\phi + \psi : A \to A$ is a homomorphism
		$(\phi+\psi)(x+y) = (\phi+\psi)(x) + (\phi+\psi)(y)$
	\begin{align}
		(\phi+\psi)(x+y) & =\phi(x+y) + \psi(x+y) \\
		& = (\phi(x)+\phi(y)) + (\psi((x)+\psi(y))\\
		& = {\color{red}\phi(x) + ((\phi(y) + \psi(x))+\psi(y))} \\
		& = {\color{red}\phi(x) + ((\psi(x) + \phi(y))+\psi(y))} \\
		& = {\color{red}(\phi(x) + \psi(x)) + (\phi(y)+\psi(y))} \\
		& = (\phi+\psi)(x) + (\phi+\psi)(y)
	\end{align}
	\end{enumerate}
\end{frame}

\begin{frame}{Step 2 : $(\phi + \psi)(x)$ is uniquely defined}
	Let $\phi,\psi \in End(A)$, and $x \in A$.
	\begin{enumerate}
		\item $\phi(x), \psi(x)$ are uniquely defined in $A$.\\
			$\forall x \in A,\ \forall \phi \in End(A),\ \exists \text{ unique } \phi(x) \in A$.
		\item Group addition is uniquely defined in $A$.\\
			{\color{red}$\phi(x), \psi(x) \in A \implies \exists \text{ unique }\phi(x) + \psi(x) \in A$}
	\end{enumerate}
\end{frame}

\begin{frame}{Step 3 : $\phi\psi \in End(A)$}
	Let $\phi,\psi \in End(A)$, and $x \in A$.
	\begin{enumerate}
		\item function $\phi\psi : A \to A$ is well-defined, $\phi\psi (x) = \phi(\psi(x))$
		\begin{enumerate}
			\item $\phi(\psi(x)) = \phi(y) \in A$ where $y = \psi(x) \in A$
			\item $\psi(x),\phi(y) \in A$ where $y \in \psi(x) \in A$.
			\end{enumerate}
		\item $\phi\psi = \phi \circ \psi$ is a homomorphism.
			\begin{align}
				(\phi\psi)(x + y) & = \phi(\psi(x+y)) \\
				& = \phi(\psi(x)+\psi(y)) \\
				& = \phi(\psi(x)) + \phi(\psi(y)) \\
				& = \phi\psi(x) + \phi\psi(y)
			\end{align}
	\end{enumerate}
\end{frame}

\begin{frame}{Step 4 : $(\phi\psi)(x)$ is uniquely defined}
	Let $\phi,\psi \in End(A)$, and $x,y \in A$.
	\begin{enumerate}
		\item $\psi(x) = y$ is uniquely defined in $A$.
		\item $\psi(x),\phi(y)$ are uniquely defined in $A$.
	\end{enumerate}

	$\forall x \in A,\ \exists \text{ unique } (\phi\psi)(x) = \phi(\psi(x)) \in A $ 
\end{frame}

\end{document}
